\section{Using The Code From This Paper}        % first appendix
%%%%%%%%%%%%%%%%%%%%%%%%%%
The code written for this paper is available at
\url{https://github.com/uga-ssrl/hare}.
The README.md must be read carefully in order to launch successfully, as with any other ROS package.

Updates to this repository will occur on a weekly basis for the foreseeable future.

\section{Attribute Listing}

Keep in mind that these were not used during this project, just designed
before changes and pivots were necessary. These attributes are to be used
as descriptors of a robots capabilities and are shared with all neighboring
robots. Many of them have associated numerical values, enumerations, or tuples
to further detail extent of related capability. All action nodes in the behavior
graph, along with some directive nodes, have requirements and target values associated
with certain attributes to direct traversal. As long as requirements are fulfilled,
the euclidean distance away from that set of target values, along with some other
specific values like location, would be used to determine the cost of the individual
robot taking that path forward in the behavior graph. If necessary, neighboring robots
would communicated their cost to one another as well as take into account what the global
objective requires to determine if one or both can take the evaluated path.

\begin{center}
  \begin{tabularx}{\textwidth}{ |X|X| }
  \hline
  Dimensionality & 0, 2D, 3D \\
  FOV & angle \\ \hline
  Global Positioning & Boolean: Good if within sensor error margin in meters \\
  Non-spatial & sensors num, telemetry type \\
  Visible Distance & In meters \\
  \hline
\end{tabularx}
\end{center}


\begin{center}
\begin{tabularx}{\textwidth}{ |X|X| }
  \hline
  Dimensionality & directional proximity, 2D, 3D \\
  Allowable Altitude Range & relative to lowest known ground 0 \\
  Terrain Compatibility &  submersible, non-submersible, ground, aerial, amphibious, submersible amphibious, full \\ \hline
  Turn Radius &  [ 0 (holonomic) , (nonholonomic) ∞) \\
  Moving appendages &  num, dexterity levels, branch/joint specifiers \\
  Motor Current Requirement &  Current (minimum) \\
  \hline
\end{tabularx}
\end{center}

\begin{center}
\begin{tabularx}{\textwidth}{ |X|X| }
\hline
Duplex & half or full \\
Transmission FOV & angle \\
Receiver FOV & angle \\
Transmitter Count & num, type specifiers \\
Receiver Count & num, type specifiers \\
Transmission S/N ratio & num \\
S/N receiving threshold & num \\
Channel & Capacitynum# \\
Range & distance \\
Bandwidth & num \\
\hline
\end{tabularx}
\end{center}

\begin{center}
\begin{tabularx}{\textwidth}{ |X|X| }
\hline
Capacity & Watt hour \\
Chargeable & no,externally, solar \\
Power profile/average draw & Voltage (low, medium, charged) \\
\hline
\end{tabularx}
\end{center}


\begin{center}
\begin{tabularx}{\textwidth}{ |X|X| }
\hline
Number of Cores &  num \\
Microcontroller(s)  & num, function specifiers \\
ASIC(s)  & num, function specifiers \\
GPU(s)  & num, function specifiers \\
FPGA(s) &  num, function specifiers \\
\hline
\end{tabularx}
\end{center}
