\subsection{Overview}

As an individual working in a cooperative operation, knowing the capabilities
of you and your teammates is necessary to the division of tasks. This idea provides
the basis for this proposed research. As homogeneous multi-robot systems are
inherently limited based on the capabilities of the specific robot, it makes
sense to venture down a heterogeneous avenue. A lot of research has gone into
heterogeneous multi-robot systems in the past few years, but a common trend
amongst most of them is that the model is built for the specific platforms
that are to be used. This proposed research is meant to provide the groundwork
for a heterogeneous model that allows for the insertion of a robots with different
skill sets with the purpose of enhancing the overall capability of the system.
This framework would allow for a variety of models to be built on top of it,
with a variety of hardware. Researchers wanting to build a specific system
could allocate funds and resources to build each member to accomplish a portion
of the overall task, possibly allowing for more cost efficiency and remove
unnecessary redundancy. The main challenge in this research will be finding
the optimal way to reassign tasks based on defined capabilities of each robot
in the system. This brings up a secondary, more basic, challenge of defining
robot’s characteristics in a simple yet comprehensive manor. All-in-all,
this research acts as a proof of concept for task division based on member
attributes, or individual robots capabilities, and will allow for a high degree
of heterogeneous utilization and multi-robot model expansion.

\subsection{Background}

The basis of this research is in the definition of attributes which is detailed
in the methodology section. This section is meant to detail previous research
that will be used in proving that the fundamental idea of task division based
on individual capabilities. As this research is meant to strictly provide a basis
for heterogeneous task division and capability utilization, the model built on
top of it could take many forms. In this case, a multi-robot behavior tree will
be constructed to guide operations and define command sequences at a high level like
proposed in [3] which is an expansion of research in [2]. To implement such a
system the DSL buzz [6] will be used which provides a level of abstraction that
allows for focus on high level cooperation, information sharing, and simple
behavioral definitions. The researchers that developed this language have created a
ROS package that will be utilized, called ROSbuzz, in this proof-of-concept system.
Control mechanisms and communication in multirobot systems have been a center of
research for a while and updates/innovation in this corner of the field are practically
constant. The Robot Operating System (ROS) \cite{ROS} provides a large variety of packages
that can be put together to form fully functioning and unique control mechanisms as
well as providing easy to use communication packages. Exploration and path finding
are widely researched and implemented using ROS packages, as components are pretty
broad and can help illuminate edge cases in multi-robot systems as agents have to
operate without global knowledge of their environment.
