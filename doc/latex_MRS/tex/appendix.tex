\appendices
\section{Using The Code From This Paper}        % first appendix
%%%%%%%%%%%%%%%%%%%%%%%%%%
The code written for this paper is available at, this repository will be updated on a periodic basis in order to
complete proposed objective.
\textbf{\url{https://github.com/uga-ssrl/hare}}.

% \section*{Acknowledgment}
\begin{table}[htbp]
\caption{Terrain index description}
\begin{center}
\begin{tabular}{|c|c|}
\hline
\textbf{Terrain}&{\textbf{Index}} \\
\cline{2-4}
\hline
Unknown & -2 \\
\hline
Wall & -1\\
\hline
Open Space & 0\\
\hline
Ramp & 1 \\
\hline
Short Tunnel & 2 \\
\hline
Tall Tunnel & 3 \\
\hline
% \multicolumn{4}{l}{$^{\mathrm{a}}$Sample of a Table footnote.}
\end{tabular}
\label{tabl:terrain}
\end{center}
\end{table}


\begin{table}[H]
\caption{Decision state description}
\begin{center}
\begin{tabular}{|c|c|}
\hline
\textbf{STATE}&{\textbf{Decision State}} \\
\cline{2-4}
\hline
IDLE& Default, low power, malfunction\\
\hline
SEARCH& Explore unknown frontier\\
\hline
RIDE& Seek goal\\
\hline
PROD& Investigation, check tunnel with unknown terrain\\
\hline
DONE& All traversable tunnels prodded, search space explored, out of goals\\
\hline
% \multicolumn{4}{l}{$^{\mathrm{a}}$Sample of a Table footnote.}
\end{tabular}
\label{tabl:tree-states}
\end{center}
\end{table}



\begin{table}[H]
\caption{Robot Characterization}
\begin{center}
\begin{tabular}{|c|c|}
\hline
\textbf{Attributes}&{\textbf{Considerations}} \\
\cline{2-4}
\hline
Dimensionality & directional proximity, 2D, 3D \\
Sight FoV & horizontal angle, vertical angle \\
\hline
GPS & error in meters\\
Telemetry & sensor input, telemetry type\\
Visible Range & distance in meters\\
\hline
Altitude & relative to state space minima\\
Safe Terrain & submersible, non-submersible, ground, aerial, amphibious, all\\
\hline
Appendages& num, type, joint specifiers, degrees of freedom\\
Tools& bucket, wedge, shovel, pike, clamp\\
Maneuverability& 3 - holonomic, 2 or 1 nonholonomic, 0\\
\hline
Communication Type & full/half duplex\\
Reciever FoV & angle horizontal, angle vertical\\
Transmitter FoV & angle horizontal, angle vertical\\
Reciever Range & meter distance\\
Transmitter Range & distance in meters\\
SNR & ratio\\
Channel(s) & number\\
Bandwidth & frequency in Hz\\
\hline
CPU & number cores, clock speed, type\\
GPU & number, clock speed, type\\
ASIC & number, type\\
FPGA & number, type\\
\hline
% \multicolumn{4}{l}{$^{\mathrm{a}}$Sample of a Table footnote.}
\end{tabular}
\label{tabl:attributes}
\end{center}
\end{table}


%
%
%
% \vspace{10cm}
%
%
%
%
% \begin{center}
% \begin{tabularx}{\textwidth}{ |X|X| }
%   \hline
%   Dimensionality & directional proximity, 2D, 3D \\
%   Allowable Altitude Range & relative to lowest known ground 0 \\
%   Terrain Compatibility &  submersible, non-submersible, ground, aerial, amphibious, submersible amphibious, full \\ \hline
%   Turn Radius &  [ 0 (holonomic) , (nonholonomic) ∞) \\
%   Moving appendages &  num, dexterity levels, branch/joint specifiers \\
%   Motor Current Requirement &  Current (minimum) \\
%   \hline
% \end{tabularx}
% \end{center}
%
% \newpage
%
% \begin{center}
% \begin{tabularx}{\textwidth}{ |X|X| }
% \hline
% Duplex & half or full \\
% Transmission FOV & angle \\
% Receiver FOV & angle \\
% Transmitter Count & num, type specifiers \\
% Receiver Count & num, type specifiers \\
% Transmission S/N ratio & num \\
% S/N receiving threshold & num \\
% Channel & Capacitynum# \\
% Range & distance \\
% Bandwidth & num \\
% \hline
% \end{tabularx}
% \end{center}
%
% \begin{center}
% \begin{tabularx}{\textwidth}{ |X|X| }
% \hline
% Capacity & Watt hour \\
% Chargeable & no,externally, solar \\
% Power profile/average draw & Voltage (low, medium, charged) \\
% \hline
% \end{tabularx}
% \end{center}
%
%
% \begin{center}
% \begin{tabularx}{\textwidth}{ |X|X| }
% \hline
% Number of Cores &  num \\
% Microcontroller(s)  & num, function specifiers \\
% ASIC(s)  & num, function specifiers \\
% GPU(s)  & num, function specifiers \\
% FPGA(s) &  num, function specifiers \\
% \hline
% \end{tabularx}
% \end{center}
